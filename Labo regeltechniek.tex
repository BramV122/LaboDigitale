\documentclass[a4paper, 12pt]{article}
\usepackage{geometry}
\usepackage{graphicx}
\usepackage{mathtools}
\usepackage[utf8]{inputenc}
\geometry{margin=1in}

\title{Laboverslag Digitale Regeltechniek}
\author{Danilo Peeters \\ Bram Verstraeten}

\begin{document}

\makeatletter
    \begin{titlepage}
	    \includegraphics[width=1\linewidth]{Logo_Uhasselt_KULeuven.jpeg}\\[30ex]
        \begin{center}
            {\huge \@title }\\[20ex] 
            {\large\@author}
        \end{center}
    \end{titlepage}
\makeatother

\newpage

\section{\underline{Opgave 1}}

\subsection{Identificeer het onbekende systeem door toepassing van de methode van Ziechler en Nichols.}

\begin{table}[!h]
\begin{large}
\centering
\resizebox{\columnwidth}{!}{
	\begin{tabular}{p{0.5\linewidth} p{0.5\linewidth}}
	$10s -> 2,5cm$ \\
	$\tau\textsubscript{v} = 0,6cm$ & $\tau = 3,7cm$ \\
	$\tau\textsubscript{v} = 0,6cm * \frac{10s}{2,5cm} = 2,4cm$ & $\tau = 3,7cm * \frac{10s}{2,5cm} = 14,8s$ \\
	$K\textsubscript{v} = 2$ \\ [2ex]
	$H(p) = \frac{2e^-2,4p}{1+14,8p}$
	\end{tabular} 
}
\end{large}
\end{table}

\subsection{Bepaal de gepaste regelaars voor een regeling hetzij storing met 20\% doorschot}

\begin{table}[!h]
\begin{large}
\centering
\resizebox{\columnwidth}{!}{
	\begin{tabular}{p{0.5\linewidth} p{0.5\linewidth}}
	$\frac{\tau}{\tau\textsubscript{v}} = \frac{14,8s}{2,4s} = 5,92$ & $3,3 < PID < 7,4 ->$ dus we gebruiken een PID regelaar \\ [3ex]
	Regelaar & Storing \\
	$K\textsubscript{R} = 0,95 * \frac{1}{K\textsubscript{p}} * \frac{\tau}{\tau\textsubscript{v}} = 2,812$ & $K\textsubscript{R} = 1,2 * \frac{1}{K\textsubscript{p}} * \frac{\tau}{\tau\textsubscript{v}} = 3,552$ \\
	$\tau\textsubscript{i} = 1,35\tau = 19,98s$ & $\tau\textsubscript{i} = 2\tau = 4,8s$ \\
	$\tau\textsubscript{d} = 0,47\tau\textsubscript{v} = 1,128s$ & $\tau\textsubscript{d} = 0,42\tau\textsubscript{v} = 1,008s$ \\
	\end{tabular}
}
\end{large}
\end{table}

\end{document}